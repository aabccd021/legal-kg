%
% Halaman Abstrak
%
% @author  Andreas Febrian @version 2.00 @edit by Ichlasul Affan
%

\chapter*{Abstrak}
\singlespacing

\vspace*{0.2cm}

\noindent \begin{tabular}{l l p{10cm}} Nama & : & \penulis \\
	Program Studi      & : & \program \\
	Judul              & : & \judul   \\
\end{tabular} \\

\vspace*{0.5cm}

\noindent Dokumen \legal pada umumnya tersedia dalam bentuk PDF yang bersifat tidak
\textit{machine-readable}, sehingga data tidak dapat diproses secara otomatis dan dalam skala besar
oleh komputer untuk dimanfaatkan dalam berbagai teknologi digital. Oleh karena itu diperlukan
struktur data yang dapat memuat informasi \legal, beserta sistem yang melakukan konversi dari PDF
menjadi struktur data tersebut. Dengan alasan tersebut, pada penelitian ini penulis mengembangkan
Lex2KG, \textit{framework} untuk mengonversi dokumen PDF peraturan perundang-undangan di Indonesia
(\textit{Lex} berasal dari Bahasa Latin yang berarti hukum) menjadi \textit{knowledge graph}.
\textit{Knowledge graph} (KG) adalah \textit{graph} yang menggambarkan entitas dunia nyata beserta
keterkaitannya dan memberikan informasi terstruktur yang \textit{machine-readable}. Pada penelitian
ini KG dipilih dari berbagai struktur data yang tersedia karena KG terkategori sebagai
\textit{5-star data} menurut \textit{5-star deployment scheme for Open Data}, yaitu data dengan
jenis informasi paling bermanfaat, memberikan data dalam bentuk \textit{open license}, terstruktur,
tersedia dalam \textit{open format}, menggunakan URI sebagai notasi data, dan dapat dihubungkan
(\textit{linked}) dengan data lain. KG \legal mengandung berbagai data terstruktur konten tekstual,
struktur dokumen, seperti metadata, serta relasi antara peraturan seperti amendemen dan rujukan.
Lex2KG memungkinkan pemanfaatan data \legal secara \textit{advanced}, otomatis, dan dalam skala
besar pada berbagai lingkup digital terutama pada industrsi hukum dan pengacara. Contoh pemanfaatan
data dapat berupa \textit{search engine}, sistem \textit{question answering}, dan analisis statistik
\legal. Menggunakan Lex2KG, penulis berhasil mengonversi 784 undang-undang menjadi KG dengan ukuran
total lebih dari 1,1 juta \textit{triple}. Salah satu peraturan yang berhasil dikonversi adalah UU
11/2020 tentang Cipta Kerja yang kontennya bersifat relatif kompleks dan berukuran besar. Penulis
juga menunjukan \textit{use case} dari KG \legal yaitu \textit{chat bot} sederhana, SPARQL
\textit{query}, dan visualisasi \legal.

\vspace*{0.2cm}

\noindent \bo{Kata kunci:} \\ \textit{Knowledge Graph}, Peraturan perundang-undangan, SPARQL, RDF,
Konversi, PDF 

\onehalfspacing
\newpage
