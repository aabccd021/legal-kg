%
% Halaman Abstract
%
% @author  Andreas Febrian @version 2.00 @edit by Ichlasul Affan
%

\chapter*{ABSTRACT}
\singlespacing

\vspace*{0.2cm}

\noindent \begin{tabular}{l l p{11.0cm}} Name & : & \penulis      \\
	Program              & : & \program      \\
	Title                & : & \judulInggris \\
\end{tabular} \\

\vspace*{0.5cm}

\noindent Most of the legal documents are available as PDF which is not machine-readable, which
means the data could not be processed automatically and in large scale by a computer to be utilized
in various digital technology. Therefore, we need a data structure that can contain a legal
information, and also a system which converts PDF into that structure. For that reason, in this
research, author developed Lex2KG, a framework wh converting legal PDF documents in Indonesia
(\textit{Lex} comes from Latin which means law) into a Knowledge Graph. A knowledge graph (KG) is a
graph that describes real-world entities and their relationships as machine-readable and structured
information, and linkable to another KG on different domain. In this research KG is choosen from
various data structure available because KG it categorized as 5-star data according to 5-star
deployment scheme for Open Data, which data comes with most beneficial information, available under
an open license, structured, open format, uses URI to denote things, and linkable to other data. The
legal KG contains various kinds of structured data such as textual content, document structures,
metadata, and relations between law such as amendments and citations. Lex2KG enables the advanced
and automatic utilization of legal data on a large scale on a various digital scope especially on
legal industry and lawyer. The utilization could be in form of search engine, question answering
system, and statistics analytics for legals. Through Lex2KG, author have successfully converted 784
Indonesian laws into a KG with a total size of over 1.1 million triples. One of the regulation that
was successfully converted was Law 11/2020 on Job Creation, which the content is relatively complex
and large. Author also shows use cases of the legal KG for simple chatbots, SPARQL querying, and
legal visualizations.

\vspace*{0.2cm}

\noindent \bo{Key words:} \\ Knowledge Graph, Law, SPARQL, RDF, Conversion, PDF 

\onehalfspacing
\newpage
