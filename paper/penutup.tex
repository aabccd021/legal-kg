%---------------------------------------------------------------
\chapter{PENUTUP}
\label{chap:6}
%---------------------------------------------------------------
Pada bab ini, Penulis akan memaparkan kesimpulan penelitian dan saran untuk penelitian berikutnya
terkait pembuatan sistem konversi Lex2KG.

%---------------------------------------------------------------
\section{Kesimpulan}
\label{sec:kesimpulan}
%---------------------------------------------------------------

Pembuatan ontologi KG \legal diawali dengan mendefinisikan \textit{competency questions}, kemudian
merancang ontologi yang dapat menjawab \textit{competency questions} tersebut. Setelah itu,
dilakukan dengan mendefinisikan entitas dengan konsep yang sama untuk dikelompokan dalam
\textit{class}, kemudian relasi antar \textit{class} tersebut dapat didefinisikan menjadi properti.
Konsep-konsep yang terdapat pada \legal seperti struktur dokumen, rujukan, amendemen, jenis
peraturan, dan metadata berhasil didefinisikan dalam \textit{class} dan properti. Setiap definisi
ontologi seperti \textit{class} dan properti serta \textit{resource} juga diberikan spesifikasi
konstruksi URI dengan \textit{namespace} yang sudah ditentukan.

Perancangan sistem konversi otomatis PDF \legal menjadi KG dapat dilakukan melalui beberapa tahap.
Perancangan sistem konversi diawali dengan ontologi KG yang akan dihasilkan untuk mengetahui data
apa saja yang perlu diekstraksi dari PDF. Kemudian memahami struktur dan konteks dari PDF untuk
merancang sistem \textit{parsing} agar dapat memperoleh data yang dibutuhkan. Variasi format PDF
yang tidak seragam dapat distandardisasi dengan melakukan OCR ulang menghasilkan PDF versi "OCR".
Kemudian PDF dipindai menjadi \textit{span} dan dibuat tiga variasi untuk mengatasi beberapa kasus
pemindaian PDF. \textit{Span} kemudian di-\textit{parsing} menjadi data terstruktur, dan
dikonstruksi menjadi KG berdasarkan ontologi yang sudah dirancang. PDF yang dapat dikonversi dengan
baik oleh Lex2KG adalah PDF yang memiliki format yang sama dengan UU yang diperoleh dari Laman
Undang-Undang Jaringan Dokumentasi dan Informasi Hukum DPR RI.

Pada penelitian ini penulis berhasil melakukan konversi untuk 784 PDF UU yang diperoleh dari Laman
Undang-Undang Jaringan Dokumentasi dan Informasi Hukum DPR RI, seperti UU 12/2019 tentang
Pertanggungjawaban Atas Pelakasanaan Anggaran Pendapatan Dan Belanja Negara Tahun Anggaran 2018,
menjadi KG dengan total lebih dari 1.1 juta \textit{triple}. Evaluasi \textit{use case} dilakukan
dalam bentuk SPARQL \textit{query}, visualisasi, dan \textit{chat bot} sederhana pada KG tersebut.
Untuk setiap \textit{competency question} disediakan SPARQL \textit{query} yang dapat menjawab
pertanyaan tersebut. Visualisasi \textit{word cloud}, statistik, \textit{graph} (dalam bentuk
\textit{node} dan \textit{vertex}) dilakukan pada KG untuk memahami KG secara lebih intuitif.
Evaluasi \textit{use case} dilakukan untuk \textit{chat bot} sederhana dengan memberikan tampilan
untuk beberapa skenario dan diberi penjelasan bagaimana \textit{chat bot} melakukan SPARQL
\textit{query} menggunakan \textit{template} sesuai skenario.

Penulis juga melakukan membuat versi
\textit{paper}\footnote{\url{https://drive.google.com/file/d/11y5ZCe7xbLMGTOpJilZNm8IVMNtF2J_6/view?usp=sharing}}
dari skripsi ini untuk di-\textit{submit} di International Semantic Web
Conference\footnote{\url{https://iswc2021.semanticweb.org/}} sebagai \textit{poster paper} dan
sedang dalam status menunggu \textit{review}.

%---------------------------------------------------------------
\section{Saran}
\label{sec:saran}
%---------------------------------------------------------------
Berdasarkan hasil penelitian ini, berikut ini adalah saran untuk pengembangan penelitian berikutnya:
\begin{enumerate}
	\item Walaupun sebagian PDF dapat di-\textit{parse} menjadi data terstruktur, pasti akan terdapat
	      kesalahan konversi. Akan lebih baik jika terdapat sistem pembenaran di mana terdapat tenaga
	      manusia secara manual dapat memberikan data yang benar, kemudian dapat digabung dengan data
	      hasil \textit{parsing}.
	\item Saat ini pembuatan KG dalam berkas Turtle dilakukan dengan menulis lengkap URI setiap
	      entitas. Akan lebih baik jika menggunakan \textit{prefix} untuk meningkatkan \textit{readability}
	      dan mengurangi ukuran berkas.
	\item Dapat dilakukan evaluasi dan \textit{feedback} dari seorang ahli hukum atau industri hukum,
				untuk dilakukan \textit{improvement} sesuai dengan kebutuhan industri yang tidak diketahui
				penulis.
\end{enumerate}
